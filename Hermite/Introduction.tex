As was already mentioned in the introduction, an important class of algorithms in computer science and structural biology deals with the exhaustive search in the 
six-dimensional space of translations and rotations of a rigid body. 

Modern exhaustive search algorithms either implement the fast 3D translational
search using the fast Fourier transform (FFT) \cite{Chacon2002,Katchalski-Katzir1992,Gabb1997,Wriggers2010,Siebert2009}
or the fast 3D rotational search by means of the spherical harmonics decomposition
and the FFT \cite{Kovacs2002} or even the fast 5D rotational search \cite{Kovacs2003,ritchie2008accelerating}.
Exhaustive search is also widely used as a preliminary step preceding the
local search or flexible refinement procedures. Thus, the quality
and the speed of the exhaustive search algorithms have a great impact on
the solution of the vast variety of problems. Therefore, we believe that new directions
of research on this topic are very important and highly valuable. 

In this section, we present the new HermiteFit algorithm that uses the orthogonal Hermite functions to perform exhaustive search in the 
six-dimensional space of rigid-body motions.
We apply this method to the problem of fitting of a high resolution X-ray structure of a protein subunit into the 
cryo-electron microscopy (cryo-EM) density map of a protein complex.
As a part of the new method, we developed an algorithm for the rotation of the decomposition in the Hermite basis and another algorithm for the conversion of
the Hermite expansion coefficients into the Fourier basis.
We demonstrate the ability of our algorithm to compete with the well--established approaches by using two examples of different difficulty, 
the PniB conotoxin peptide and the GroEL complex. 
%
The first example illustrates encoding principles and demonstrates the influence of the encoding quality on the goodness of  fit. 
%
The second example is the gold standard of all electron density map fitting 
algorithms. 
%
Our approach allows to analytically assess the quality of encoding of the Hermite basis using an estimation of the crystallographic R-factor. 
We then compare this estimation with the one computed numerically for the PniB conotoxin density map.
%
Finally, we compare the speed and the fitting accuracy of our algorithm with the two popular programs, the ADP\underline{\hspace*{0.2cm}}EM fitting method and the {\em colores} program from the 
Situs package and demonstrate that HermiteFit spends less running time per one search point compared to the two other methods while attaining a similar accuracy.

The HermitFit algorithm can be straightforwardly applied to a broad class of problems in different fields of research.
For example, one of the bottlenecks of the algorithms for molecular replacement in crystallography is the
computation of the Fourier coefficients (structure factors) of a molecule \cite{navaza1995fast}. 
This operation is to be precise and fast. However, the exact analytical evaluation 
of the structure factors is too costly \cite{sayre1951calculation} when recomputing them for each rotation of the molecule. 
Therefore, currently one uses the Sayre--Ten Eyck approach to compute the Fourier 
coefficients \cite{ten1977efficient}. Unfortunately, one has to be very careful tuning the parameters of the electron density model and
the grid cell size to obtain the desired precision \cite{navaza2002computation,afonine2003fast}.
Unlike the Sayre--Ten Eyck, our algorithm offers the analytical expression for the structure factors of the Hermite decomposition of a molecule.
%
Finally, our approach  allows to analytically estimate the quality of encoding using, e.g., crystallographic R-factors.