In the previous section we described how coefficients of the Hermite decomposition change upon rotation of the density. However all intrinsic physical properties of the 
protein should not change upon rotation and translation of the reference frame. Therefore a description of the density that is invariant under rotation and translation 
is of great interest.

The rotational invariants originally were employed in 3D object recognition by Hue et all. \ref{hu1962visual}. To construct them they used geometrical moments:
$$M_{ijk}=\int\int\, f(x_{1},x_{2},x_{3})x_{1}^{i}x_{2}^{j}x_{3}^{k}\, dx_{1}\, dx_{2}\, dx_{3}$$
where $f$ stands for the density described and the integration is done over all 3D space where $f$ is defined.

After this peoneering work, much effort had been put into constructing invariants of geometric moments upon rotation, 
translation, general affine transformations and even projection operators. So far, several methods are available to 
construct systems of geometrical invariants \ref{hickman2012geometric, diao2009linear}. They are 
sucessfully applied in 2D pattern recognistion and 3D object recognition \ref{flusser2009moments} and positioning of 3D objects \ref{taubin1991recognition}.

However in many cases these moments are not the best choise. Other types of moments been investigated such as Legendre moments \ref{hosny2010new}, 
Zernike moments \ref{venkatraman2009protein} and others\ref{mak2008extension}. The Hermite-Gaussian moments haven't been previously explored in 3D, but they showed to be very 
useful in 2D image reconstruction \ref{yang2011image,rahman2013low}. In this section we provide a method to build 3D Hermite invariants.

First, by analogy with geometric invariants let's introduce Gaussian-Hermite invariants:
$$m_{pqr}=\int\int\int f(x,y,z)\psi_{p}(x)\psi_{q}(y)\psi_{r}(z)\, dx\, dy\, dz$$

The derivation of Hermite-Gaussian moments invariants is based on the exsting geometric moments invariants. Let's denote operator of rotation in $R^{3}$ as $\mathbf{R}$:
$r'=\mathbf{R}r$.
Geometric rotational invariants are algebraic functions of geometric moments that do not change their values after application of $\mathbf{R}$ operator. 
An example of such a function is:
$$I_{1}=\mu_{200}+\mu_{020}+\mu_{002}$$
which written in terms of integrals sums up to:
$$I_{1}=\int\int\int f(x,y,z)(x^{2}+y^{2}+z^{2})\, dx\, dy\, dz$$
The value of $I_{1}$ is simply a mass of the function, that is invariant upon rotations.
Let's now look closer at the integral form of Gaussian-Hermite moments:
$$m_{pqr}\propto \int\int\int f(x,y,z)e^{-(x^{2}+y^{2}+z^{2})/2\sigma^{2}}H_{p}(\frac{x}{\sigma})H_{q}(\frac{y}{\sigma})H_{r}(\frac{z}{\sigma})\, dx\, dy\, dz$$
Where the coefficient of proportionality does not depend on $x,y,z$ and therefore can be ommitted. The factor in the exponent is also invariant with
respect to the rotations and dos not influence the rotational behaviour of Hermite polynomials. Further we ommit these two factors.

Hermite polynomials can be further expanded in terms of $\frac{x}{\sigma}$, $\frac{y}{\sigma}$ and $\frac{z}{\sigma}$ and therefore each Hermite moment 
can be rewritten as the sum of monomials. For example:
$$m_{000} \propto  \int f(x,y,z) \, dx\, dy\, dz$$
$$m_{111} \propto  \int f(x,y,z) \left( 1 + 2\frac{x}{\sigma} \right)\left( 1 + 2\frac{y}{\sigma}\right)\left( 1 + 2\frac{z}{\sigma}\right)\, dx\, dy\, dz$$
%$$m_{222} \propto  \int f(x,y,z) \left( -1 + 2\frac{x}{\sigma} +4\left(\frac{x}{\sigma}\right)^2 \right)\left( -1 + 2\frac{y}{\sigma} +4\left(\frac{y}{\sigma}\right)^2 \right)\left( -1 + 2\frac{z}{\sigma} +4\left(\frac{z}{\sigma}\right)^2 \right)\, dx\, dy\, dz$$
Easy to see that $m_{111}$ behaves similar to the following combination of geometric moments:
$$m_{111}\propto \mu_{000} + 2\left(\mu_{100}+\mu_{010}+\mu_{001}\right) + 4\left(\mu_{110}+\mu_{011}+\mu_{101}\right) + 8\mu_{111}$$
On the other hand suppose we know several geometric invariants $I^(k) = f^k (\mu_{000},\mu_{100},\ldots)$. We now can eliminate variables $\mu_{ijk}$ in each
geometric invariant and obtain the Gaussian-Hermite invariants.

Summing up, the algorithm for deriving Gauss-Hermite invariants from geometric ones is:
\begin{enumerate}
 \item Get $k$-th geometric invariant $I_{k}^{G}(\mu_{000},...,\mu_{PQR})$
 \item From equations $m_{lmn}=M_{lmn}(\mu_{000},\ldots,\mu_{lmn}),~~ l=0\ldots L,~~ m=0 \ldots M,~~ p=0\ldots P$
  derive expressions for $\mu_{lmn}=M_{lmn}(m_{000},\ldots,m_{lmn}),~~ l=0\ldots L,~~ m=0 \ldots M,~~ p=0\ldots P$,
  where $P,Q,R \leq K$
 \item Substitute $\mu_{lmn}$ in geometric invariant $I_{k}$ for $M_{lmn}$
\end{enumerate}

The second step of the algorithm is valid because from the definition of Hermite polynomials the powers in products of $xyz$ do not exceed the order of moment. 
And therefore we have system of $K$ equations dependent on $K$ variables.

 