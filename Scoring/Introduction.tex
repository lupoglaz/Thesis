As was described in the introduction, scoring method is used to filter out false-positive predictions of rigid-body docking step and refine those close to the
native conformation of a protein-protein complex. Therefore the method of scoring has a decisive role in success of the whole docking workflow. 

In this section
we propose a new method to derive scoring functions. We base our method on separation of the native structures and computationally generated non-native conformations of a complex (decoys). Most of
previously used algorithms solving this problem separate all the decoys from all the decoys simultaneously. The key new idea behind our algorithm is that the decoys of 
a particular complex should be separated from its native structure only. However the form of the scoring potentials should be the same for all the dataset. Based on these
prepositions we show that this problem leads to the well-defined convex quadratic optimization problem. We measure the performance of the scoring functions obtained 
on the commonly used benchmarks. We show that our algorithm has inherent stability against overfitting. Also, due to the properties of the basis we used our scoring 
potentials show some interesting coarse-graining properties.

Given the widely recognized importance of water molecules at the protein-protein interface we also developed potentials 
for the prediction of water molecules relying on the general ideology of the scoring potentials we developed for the 
protein-protein contacts.