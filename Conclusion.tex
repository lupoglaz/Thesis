The aim of the work was the development and validation of new algorithms that could lead to certain advances in the fields of exhaustive rigid-body search
and scoring of protein-protein complexes conformations. During this work the Hermite fitting algorithm was proposed that is not only competable with
the state-of-art approaches to this problem, but adds a new class of algorithms, that operate in Hermite functions space, to the existing ones, that 
function if spherical harmonics and grid representations of a function. A new algorithm to obtain a scoring functions was proposed that 
is derived from basic logical considerations about the nature of the training dataset. It avoids common unsolved problems with the reference state
and has a valuable property of global convergence. It was applied with success to the problems of protein-protein conformations scoring as well 
as to the prediction of positions of crystallographic water molecules at the protein-protein interaction interface. It was validated using 
well established benchmarks and community-wide critical prediction of protein interaction assessement challenge.

\subsection{Future developments}
With the advent of the post-genomic era, the cost of whole-genome sequencing plummets and the number of sequenced organisms grows rapidly. However,
the proteomics field still did not step into the phase of the exponential growth. Therefore I believe, that probing protein-protein interactions 
on the scale of proteome will revolutionize the fields of interactomics, evolution and systems biology. The methods that currently applied to 
discover protein interaction network are either large-scale, but give big number of false- positives and negatives or fit for the discovering 
detailed picture of single protein-protein complex. Bridging the gap between these two classes of methods would be a tremendous leap forward 
in the field of protein-protein interaction prediction.

Therefore it is worth trying to optimize current rigid-body search algorithms to scale the computations up. The parallel computing paradigm and
especially general GPU programming could lead to a considerable HermiteFit algorithm runtime reduction.

In the area of scoring functions I believe that the paradigm of predefined atom types should be overcomed. The newly emerging area of dimentionality
reduction and deep learning surely will bring new advances to the scoring field.

Another direction of expanding this work is to integrate the developed algorithms into one user-friendly package for the convenient use by those who do not
posess programming skills required to implement them.