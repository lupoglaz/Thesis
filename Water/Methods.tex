\subsubsection{Theory}

We assume that the solvation free energy of a protein can be approximated via a linear functional $F(n(r))$, which depends on  the distribution of distances $n(r)$ between the interaction sites $k$ on the protein and water molecules around it,
of the following form:
\begin{eqnarray}
\label{eq:functional}
F(n(r))\equiv F(n^{1}(r),..,n^{M}(r)) = \sum_{k=1}^M  \int \limits_{0}^{r_{max}} n^{k}(r)U^{k}(r)~dr, 
\label{eq:solvationFun}
\end{eqnarray}
where $r_{max}$ is the protein -- water cutoff interaction distance, $M$ is the number of discrete interaction sites on the protein that are located at the sites of the atomic nuclei, $n^{k}(r)$ are  the number densities of $k$th protein interaction site -- water oxygen pairs at distance $r$, and $U^{k}(r)$ are the unknown  \emph{solvation scoring functions}  that are to be determined from the training set of native protein complexes. 
Once we determine the solvation scoring functions $U^{k}(r)$, we can compute the value of the solvation energy functional $F(n(r))$ given the positions of the protein and the surrounding water molecules.
%
Then, we minimize  $F(n(r))$ with respect to protein -- water distribution functions $n(r)$. The  distribution functions of the minimized functional
will then correspond to the positions of the native water molecules around the protein.

Next, we explain how we determine the unknown solvation scoring functions $U^{k}(r)$. Given a set of orthogonal functions $\xi (r)$ defined on the interval of orthogonality $[0;r_{max}]$ with a unit weight function such that
\begin{eqnarray}
\int \limits_{0}^{r_{max}} \xi_{i}(r) \xi_{j}(r)~dr = \delta_{i j}
\label{eq:orth}
\end{eqnarray}
where $\delta_{i j}$ is the Kronecker delta function, we expand the solvation scoring functions $U^{k}(r)$ and the number densities $n^{k}(r)$ on the interval $[0;r_{max}]$ as

\begin{eqnarray}
\label{eq:expU}
U^{k}(r)=\sum_i w_i^{k} \xi_i (r) ,~~~~r \in[0;r_{max}] \\
n^{k}(r)=\sum_i x_i^{k} \xi_i (r) ,~~~~r \in[0;r_{max}]
\label{eq:expg}
\end{eqnarray}

Expansion coefficients $w_i^{k}$ and $x_i^{k}$ can be determined from the orthogonality condition (\ref{eq:orth}) as 
\begin{eqnarray}
w_i^{k} = \int \limits_{0}^{r_{max}}U^{k}(r) \xi_i (r)~dr \\
x_i^{k} = \int \limits_{0}^{r_{max}}n^{k}(r) \xi_i (r)~dr
\label{eq:projection}
\end{eqnarray}
Finally, 
we expand the functional $F(n(r))$ up 
to order $P$  as

\begin{eqnarray}
\label{eq:expansion}
F(n(r)) \approx \sum_{k=1}^M \sum_{i}^{P} w_{i}^{k}x_{i}^{k} = (\mathbf{w} \cdot \mathbf{x}),~~~~\mathbf{w}, \mathbf{x} \in \mathbb{R}^{P \times M}
\end{eqnarray}

In order to determine the unknown functions  $U^{k}(r)$, we require that for each protein complex $i$ from the training dataset,  $i=1...N$, the solvation functional 
(\ref{eq:expansion}) takes the minimum value only for native protein -- water distributions $n^{k}_{i0}(r)$. Therefore, for each protein complex $i$, we generate $D$ 
non-native protein -- water distributions $n^{k}_{ij}(r)$, $j=1...D$,
and
formulate the following  soft-margin quadratic optimization problem with respect to the solvation scoring function expansion coefficients $\mathbf{w}$:

\begin{eqnarray}
\begin{tabular}{lc}
Minimize (in  $\mathbf{w}$, $b_j$, $\eta_{ij}$):& $\frac{1}{2} \mathbf{w}\cdot\mathbf{w}+ \sum_{ij}C_{ij}\eta_{ij}$ \\ 
\multirow{2}{*}{Subject to:}
&$y_{ij} \left[ \mathbf{w}\cdot \mathbf{x}_{ij}-b_j \right]-1 +\eta_{ij} \geq 0, ~ i=1...N,~j=0...D$ \\
&$\eta_{ij} \geq 0$
\end{tabular}
\label{eq:svmSoftMargin}
\end{eqnarray}
%
where parameters $y_{i0} = -1$ for native protein-water distributions $\mathbf{x}_{i0}$ and $y_{ij} = 1$ otherwise. 
Parameters $C_{ij}$ can be regarded as regularization parameters.  We set $C_{i0} = C D/(D+1)$ for native protein-water distributions $\mathbf{x}_{i0}$ 
and $C_{ij} = C/(D+1)$ otherwise, where we determine the best value for $C$  from the training set. More details can be found in \cite{Derevyanko4}.
